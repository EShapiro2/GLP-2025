\subsection{NAT Traversal Support}

In typical mobile deployments, GLP agents are often behind NATs or firewalls that block incoming connections. To establish direct communication, agents can use standard Internet protocols~\cite{ford2006p2pnat,chowdhury2020nat}:

\begin{itemize}
    \item \textbf{STUN} (Session Traversal Utilities for NAT) allows agents to discover their public IP address and port mappings.
    \item \textbf{TURN} (Traversal Using Relays around NAT) enables fallback relaying of encrypted messages when direct peer-to-peer paths are unavailable.
    \item \textbf{TURNS}, the secure variant of TURN, ensures relay traffic is encrypted in transit.
\end{itemize}

GLP agents may attempt peer-to-peer connectivity using STUN-based discovery and fall back to TURN relays when needed. Since all GLP messages are encrypted and authenticated end-to-end at the application layer, relay servers handle only opaque ciphertext. Multiple STUN and TURN servers can coexist and be independently operated.