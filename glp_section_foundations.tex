\section{Mathematical Foundations of Grassroots Platforms}\label{sec:foundations}

This section presents the mathematical foundations for grassroots platforms, following~\cite{shapiro2025atomic}.

\subsection{Transition Systems}

\begin{definition}[Transition System, Computation, Run]\label{definition:ts}
A \temph{transition system} is a tuple $TS=(S,s0,T)$, where: 
\begin{enumerate}
    \item $S$ is an arbitrary non-empty set, referred to as the set of \temph{states}.
    \item Some $s0\in S$ is designated the \temph{initial state}. 
    \item $T\subset S^2$ is a set of \temph{transitions over} $S$, where each transition $t\in T$ is a pair $(s,s')$ of non-identical states $s\ne s'\in S$, also written as $t=s\rightarrow s'$.
\end{enumerate} 
A \temph{computation} of $TS$ is a (nonempty, potentially infinite) sequence of states $r= s_1,s_2,\cdots$ such that for every two consecutive states $s_i,s_{i+1} \in r$, $s_i\rightarrow s_{i+1} \in T$. If $s_1=s0$ then the computation is called a \temph{run} of $TS$.
\end{definition}

\subsection{Atomic Transactions and Multiagent Transition Systems}

We assume a potentially infinite set of \emph{agents} $\Pi$, but consider only finite subsets of it, so when we refer to a particular set of agents $P \subset \Pi$ we assume $P$ to be nonempty and finite. We use $\subset$ to denote the strict subset relation and $\subseteq$ when equality is also possible.

We use $S^P$ to denote the set $S$ indexed by the set $P$, and if $c\in S^P$ we use $c_p$ to denote the member of $c$ indexed by $p\in P$. Intuitively, think of such a $c\in S^P$ as an array of cells indexed by members of $P$ with cell values in $S$.

\begin{definition}[Local States, Configuration, Transaction, Participants]\label{definition:at}
Given agents $Q \subset \Pi$ and an arbitrary set $S$ of \temph{local states}, a \temph{configuration} over $Q$ and $S$ is a member of $C:= S^Q$. An \temph{atomic transaction}, or just \emph{transaction}, over $Q$ and $S$ is any pair of configurations $t=c\rightarrow c' \in C^2$ such that $c\ne c'$, with $t_p := c_p \rightarrow c'_p$ for any $p\in Q$, and with $p$ being an \temph{active participant} in $t$ if $c_p\ne c'_p$, \temph{stationary participant} otherwise.
\end{definition}

\begin{definition}[Degree]\label{definition:degree}
The \temph{degree} of a transaction $t$ (unary, binary, \ldots, $k$-ary) is the number of active participants in $t$, and the \temph{degree} of a set of transactions $T$ is the maximal degree of any $t\in T$.
\end{definition}

\begin{definition}[Multiagent Transition System]\label{definition:mts}
Given agents $P \subset \Pi$ and an arbitrary set $S$ of \temph{local states} with a designated \temph{initial local state} $s0\in S$, a \temph{multiagent transition system} over $P$ and $S$ is a transition system $TS= (C,c0,T)$ with \temph{configurations} $C:= S^P$, \temph{initial configuration} $c0:= \{s0\}^P$, and \temph{transitions} $T\subseteq C^2$ being a set of transactions over $P$ and $S$, with the \temph{degree} of $TS$ being the degree of $T$.
\end{definition}

Rather than specifying a multiagent transition system over a set of agents $P$ directly, we specify it via atomic transactions, which are typically of bounded degree smaller than $|P|}$.

\begin{definition}[Transaction Closure]\label{definition:closure}
Let $P\subset \Pi$, $S$ a set of local states, and $C:=S^P$.
For a transaction $t=(c\rightarrow c')$ over local states $S$ with participants $Q \subseteq P$, the \temph{$P$-closure of $t$}, $t{\uparrow}P$, is the set of transitions over $P$ and $S$ defined by:
$$
t{\uparrow}P := \{ t' \in C^2 :
\forall q\in Q.(t_q = t'_q) \wedge \forall p\in P\setminus Q.(p\text{ is stationary in }t')\}
$$
If $R$ is a set of transactions, each $t\in R$ over some $Q\subseteq P$ and $S$, then the \temph{$P$-closure of $R$}, $R{\uparrow}P$, is the set of transitions over $P$ and $S$ defined by:
$$
R{\uparrow}P := \bigcup_{t\in R} t{\uparrow}P
$$
\end{definition}

Namely, the closure over $P\supseteq Q$ of a transaction $t$ over $Q$ includes all transitions $t'$ over $P$ in which members of $Q$ do the same in $t$ and in $t'$, and the rest remain in their current (arbitrary) state.

\begin{definition}[Transactions-Based Multiagent Transition System]\label{definition:tbmts}
Given agents $P \subset \Pi$, local states $S$ with initial local state $s0\in S$, and a set of transactions $R$, each $t\in R$ over some $Q\subseteq P$ and $S$, the \temph{transactions-based multiagent transition system} over $P$, $S$, and $R$ is the multiagent transition system $TS= (S^P,\{s0\}^P,R{\uparrow}P)$.
\end{definition}

In other words, one can fully specify a multiagent transition system over $S$ and $P$ simply by providing a set of atomic transactions over $S$, each with participants $Q\subseteq P$.

\subsection{Protocols and Grassroots Protocols}

A protocol is a family of multiagent transition systems, one for each set of agents $P\subset \Pi$, which share an underlying set of local states $\calS$ with a designated initial state $s0$.

\begin{definition}[Local-States Function]\label{definition:local-states-function}
A \temph{local-states function} $S: 2^\Pi \mapsto 2^\calS$ maps every set of agents $P \subset \Pi$ to a set of local states $S(P) \subset \calS$ that includes $s0$ and satisfies $P \subset P' \subset \Pi \implies S(P) \subset S(P')$.
\end{definition}

Given a local-states function $S$, we use $C$ to denote configurations over $S$, with $C(P) := S(P)^P$ and $c0(P):= \{s0\}^P$.

\begin{definition}[Protocol]\label{definition:protocol}
A \temph{protocol} $\calF$ over a local-states function $S$ is a family of multiagent transition systems that has exactly one transition system $\calF(P) = (C(P),c0(P),T(P))$ for every $P \subset \Pi$.
\end{definition}

\begin{definition}[Projection]\label{definition:projection}
Let $\emptyset \subset P \subset P' \subset \Pi$.
If $c'$ is a configuration over $P'$ then $c'/P$, the \temph{projection of $c'$ over $P$}, is the configuration $c$ over $P$ defined by $c_p := c'_p$ for every $p\in P$.
\end{definition}

Note that in the definition above, $c_p$, the state of $p$ in $c$, is in $S(P')$, not in $S(P)$, and hence may include elements ``alien'' to $P$, e.g., logic variables shared with $q\in P'\setminus P$.

\begin{definition}[Oblivious, Interactive, Grassroots]\label{definition:grassroots}
A protocol $\calF$ is:
\begin{enumerate}
    \item \temph{oblivious} if for every $\emptyset \subset P \subset P' \subseteq \Pi$, 
    $T(P){\uparrow}P'\subseteq T(P')$
    \item \temph{interactive} if for every $\emptyset \subset P \subset P' \subseteq \Pi$ and every configuration $c\in C(P')$ such that $c{/}P\in C(P)$, there is a computation $c\xrightarrow{*} c'$ of $\calF(P')$ for which $c'{/}P\notin C(P)$.
    \item \temph{grassroots} if it is oblivious and interactive.
\end{enumerate}
\end{definition}

Being oblivious implies that if a run of $\calF(P')$ reaches some configuration $c'$, then anything $P$ could do on their own in the configuration $c'/P$ (with a transition from $T(P)$), they can still do in the larger configuration $c'$ (with a transition from $T(P')$), effectively being oblivious to members of $P'\setminus P$.

Being interactive is a weak liveness requirement: no matter what members of $P$ do, if they run within a larger set of agents it is always the case that they can eventually interact with non-$P$'s in a way that leaves ``alien traces'' in the local states of $P$, so that the resulting configuration $c'/P$ could not have been produced by $P$ running on their own.

\subsection{Transactions-Based Protocols are Grassroots}

\begin{definition}[Transactions Over a Local-States Function]\label{definition:transactions-lsf}
Let $S$ be a local-states function. A set of transactions $R$ is \temph{over $S$} if every transaction $t\in R$ is a multiagent transition over $Q$ and $S(Q')$ for some $Q \subseteq Q'\subset \Pi$. Given such a set $R$ and $P\subset \Pi$,
$R(P) := \{ t\in R : t \text{ is over } Q \text{ and } S(Q'), Q \subseteq Q'\subseteq P\}$.
\end{definition}

\begin{definition}[Transactions-Based Protocol]\label{definition:tb-protocol}
Let $S$ be a local-states function and $R$ a set of transactions over $S$.
Then a \temph{protocol $\calF$ over $R$ and $S$} includes for each set of agents $P\subset \Pi$ the transactions-based multiagent transition system $\calF(P)$ over $P$, $S(P)$, and $R(P)$: $\calF(P) := (S(P)^P,\{s0\}^P,R(P){\uparrow}P)$.
\end{definition}

\begin{proposition}[Transactions-Based Protocols are Oblivious]\label{proposition:oblivious}
A transactions-based protocol is oblivious.
\end{proposition}

\begin{definition}[Interactive Transactions]\label{definition:interactive-transactions}
A set of transactions $R$ over a local-states function $S$ is \temph{interactive} if for every $\emptyset \subset P \subset P' \subset \Pi$ and every configuration $c\in C(P')$ such that $c/P\in C(P)$, there is a computation $(c\xrightarrow{*} c')\subseteq R(P'){\uparrow}P'$ for which $c'/P\notin C(P)$.
\end{definition}

\begin{theorem}[Interactive Transactions Induce Grassroots Protocols]\label{theorem:interactive-grassroots}
A protocol over an interactive set of transactions is grassroots.
\end{theorem}

\begin{proof}
Let $\calF$ be a protocol over a set of transactions $R$, where $R$ is interactive.
Since $\calF$ is a transactions-based protocol then, according to Proposition~\ref{proposition:oblivious}, $\calF$ is oblivious. And since $\calF$ is over an interactive set of transactions, $\calF$ is interactive.
Therefore, by Definition~\ref{definition:grassroots}, $\calF$ is grassroots.
\end{proof}
